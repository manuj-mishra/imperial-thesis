\chapter{Preliminaries} \label{preliminaries}

\section{Cellular Automata}

A cellular automaton (CA) is a computational model that performs multiple parallel computations, each depending on local interactions, to produce complex global behaviour. We define a CA formally as follows.

\begin{definition}[Cellular automaton]
A cellular automaton is an $n$-dimensional finite grid of computational units called cells. Each cell $c_i$ is characterised by:
\begin{itemize}
  \item A discrete state variable $\sigma_i(t) \in \Sigma$, where $i$ indicates the index of the cell in the lattice, $t$ indicates the current time step, and $\Sigma$ denotes the finite set of all state variables.
  \item A finite local neighbourhood set $\mathcal{N}(c_i)$ with cardinality $N$.
  \item A transition function $\phi:\Sigma^N \to \Sigma$ which takes local neighbour states as input. This is also known as the CA "update rule".
\end{itemize}

At each time step, the state of each cell is simultaneously updated according to the transition function. That is, $\sigma_i(t+1) = \phi( \:\{ \sigma_j(t) \: | \: c_j \in \mathcal{N}(c_i) \} \:)$
\end{definition}

Due to the breadth of systems studied in CA literature, the constraints of this definition are often altered to produce interesting arrangements. For example:
\begin{itemize}
  \item The structure need not be a square grid. CA have been studied on hexagonal grids\cite{encinas2007modelling}, aperiodic tessellations such as as the Penrose tiling\cite{goucher2012gliders}, and even randomly generated structures like the Voronoi partition\cite{shi2000development}.
  \item The system need not be deterministic. Probabalistic cellular automata (PCA) have stochastic transition functions which describe a probability distribution of possible outcomes for any given input. PCA are able to model random dynamical systems in the real world from stock markets\cite{bartolozzi2004stochastic} to infectious diseases\cite{mikler2005modeling}.
  \item The state space $\Sigma$ need not be finite. In this thesis we will explore multiple possible state variable representations including bit arrays and continuous vectors.
\end{itemize}
For the purpose of this thesis, we will assume the original definition of CA unless otherwise stated.

\subsection{Neighbourhood Functions}
We consider a "neighbourhood function" for each cell $c_i \mapsto \mathcal{N}(c_i)$. This makes it easier to discuss neighbourhood sets of cells in the CA, each of which are typically homogenous.\\

There are many possible neighbourhood functions for any given CA geometry. When defining the neighbourhood function, we select a distance metric $d:\mathbb{R}^n \times \mathbb{R}^n \to \mathbb{R}$ to measure the proximity of two cells and we set a threshold $T$ under which we consider two cells to be within each other's neighbourhood.
\[c_i \in \mathcal{N}(c_j) \iff d(c_i, c_j) \leq T\]

There are two neighbourhoods that are frequently used on Euclidean lattices. The \textit{von Neumann neighbourhood} contains all cells within a Manhattan distance of 1. For a 2D square lattice, this contains the cell itself and the 4 cells in the cardinal directions. For a 3D cubic lattice, it contains the central cell and a 6-cell octahedron around it. The \textit{Moore neighbourhood} contains all cells at a Chebyshev distance of 1. For a 2D square lattice, this is the central cell with the 8 neighbouring cells in a square around it. In the 3D case, it is a cube.

\begin{figure}[!h]
\centering
\includegraphics[width=.3\textwidth]{images/neighbourhoods.png}
\caption{(a) von Neumann Neighbourhood and (b) Moore neighbourhood on a 2D square lattice \cite{debasis2011survey}}
\label{fig:neighbourhoods}
\end{figure}

In a finite grid CA, border cells must be given special consideration since they do not have the same number of neighbours as interior cells and therefore cannot share the same neighbourhood function. One option is to define a case-wise neighbourhood function with different behaviour for border cells. Another option is to freeze the state of border cells. In the field of partial differential equations, this is known as setting "fixed boundary conditions".  The problem can also be circumvented entirely by relaxing the finite grid assumption and allowing cells to "wrap around" the grid. This is known as setting "periodic boundary conditions" and can be imagined visually as running the CA on an infinite periodic tiling or, alternatively, on a torus.\\


\subsection{Conway's Game of Life}
A popular example of a CA is the Game of Life (henceforth "Life") formulated by John Conway in 1970 \cite{gardner1970fantastic}. It consists of a 2D grid of cells, each with a boolean state variable signifying that the cell is either "alive" or "dead". The transition rule takes as input the cell's own state $\sigma_i(t)$ and the number of living individuals in the cell's Moore neighbourhood (excluding itself), denoted $n$. This is as follows:

\begin{equation}
  \phi(\sigma_i(t), n) = 
\begin{cases}
  0 & \sigma_i(t) = 1 \text{ and } n < 2 \text{  (Death by "exposure")}\\
  0 & \sigma_i(t) = 1 \text{ and } n > 3 \text{  (Death by "overcrowding")}\\
  1 & \sigma_i(t) = 1 \text{ and } n \in \{2,3\} \text{  (Survival)}\\
  1 & \sigma_i(t) = 0 \text{ and } n = 3 \text{  (Resurrection)}\\
  0 & \text{otherwise}
\end{cases}
\end{equation}

Despite its simple setup and update rule, Life can exhibit the emergence of complex patterns. It is possible to simulate a fully universal Turing machine within Life [CITE] and, as a corollary of the Halting Problem [CITE], this means that Life is undecidable. Given two configurations, it is impossible to algorithmically determine whether one will follow the other.\\

Patterns found within Life include still lifes like the \textit{block} which are fixed-point solutions to the transition function as well as periodic oscillators like the \textit{beacon} which has period 2. There are also periodic patterns that move across the lattice such as the \textit{glider} pattern. It is possible to discover new stable patterns by repeatedly running specific rules on random initial patterns of a pre-determined density (called soups) and classifying the objects remaining after transient reactions have dissipated. Large-scale experiments of this nature are called "soup searches"[CITE].\\

\begin{figure}[!h]
\centering
\includegraphics[width=0.8\textwidth]{images/life-glider.png}
\caption{The glider pattern in the Game of Life \cite{dorin2012framework}}
\label{fig:life-glider}
\end{figure}

A CA is considered "Life-like" if it exists on a 2D lattice, has binary state, uses the Moore neighbourhood function. Life-like cellular automata exist in two varieties: inner-totalistic and outer-totalistic.

\begin{definition}[Inner-totalistic]
A Life-like CA is inner-totalistic if the output of the transition function depends only on the number of living cells in a cell's neighbourhood (including the cell itself).
\[
  \sigma_i(t+1) = \sigma_j(t+1) \iff \sum_{\mathclap{c_p \in \mathcal{N}(c_i)}}\sigma_p(t) = \sum_{\mathclap{c_q \in \mathcal{N}(c_j)}}\sigma_q(t)
\]
\label{def:inner-totalistic}
\end{definition}

\begin{definition}[Outer-totalistic]
A Life-like CA is outer-totalistic if the output of the transition function depends on both the number of living cells in a cell's neighbourhood and the state of the cell itself.
\[
  \sigma_i(t+1) = \sigma_j(t+1) \iff \sum_{\mathclap{c_p \in \mathcal{N}(c_i)}}\sigma_p(t) = \sum_{\mathclap{c_q \in \mathcal{N}(c_j)}}\sigma_q(t) \quad \textnormal{and} \quad \sigma_i(t) = \sigma_j(t) 
\]
\label{def:outer-totalistic}
\end{definition}

As an example of the subtle difference here, consider the configurations shown in Figure~\ref{fig:two-moores}. An inner-totalistic CA would yield identical configurations in the next time step since both input configurations have 3 active cells in the neighbourhood set. However, an outer-totalistic CA would treat both configurations separately as one has an live centre cell and the other has a dead centre cell. This discrepancy corresponds to a great difference in the size of search spaces. There are $2^{10}=1024$ inner-totalistic CA but $2^{18} = 262144$ outer-totalistic CA. A B/S rulestring represents the transition function of an outer-totalistic CA in a form called birth-survival notation.\\

\begin{figure}[!h]
  \centering
  \begin{minipage}{.4\textwidth}
    \centering
    \includegraphics[width=.4\linewidth]{images/moore_1.png}
  \end{minipage}%
  \begin{minipage}{.4\textwidth}
    \centering
    \includegraphics[width=.4\linewidth]{images/moore_2.png}
  \end{minipage}
  \caption{Two possible configurations of a Life-like CA[CITE]}
  \label{fig:two-moores}
\end{figure}

\begin{definition}[Birth-survival notation]
Let $N_b$ and $N_s$ be sets of integers. We say an outer-totalistic CA has rulestring \textnormal{B$N_b$/S$N_s$} if it has transition function:
\[
  \phi(\sigma_i(t), n) = 
  \begin{cases}
    1 & \sigma_i(t) = 0 \text{ and } n \in N_b \text{  (Birth)}\\
    1 & \sigma_i(t) = 1 \text{ and } n \in N_s \text{  (Survival)}\\
    0 & \text{otherwise}
  \end{cases}
\]
\label{def:bs-notation}
\end{definition}

Using this notation, we can represent the Game of Life as B3/S23. In this thesis, when we refer to Life-like CA, we implicitly assume the outer-totalistic variety.


\subsection{Wolfram's Classification}

The choices of lattice geometry, neighbourhood function, state variable, and transition rule define the behaviour of a CA. Fixing the former three factors, Wolfram \cite{wolfram1986theory} classified CAs based on transition rules as follows:
\begin{enumerate}
  \item Class 1 (Null) : Rules that lead to a trivial, uniform state
  \item Class 2 (Fixed-point / Periodic) : Rules that lead to stable or periodic patterns
  \item Class 3 (Chaotic) : Rules that lead to chaotic patterns
  \item Class 4 (Complex) : Rules that lead to complex, long-lived impermanent patterns
\end{enumerate}

\textbf{Elementary cellular automata} are defined on the simplest nontrivial lattice, a finite one-dimensional chain. The neighbourhood of each cell contains the cell itself and the two cells adjacent to it on either side. The state variable is a boolean which means there are $2^3 = 8$ possible neighbourhood state configurations. A transition rule maps each of these neighbourhood states to a resultant state and can therefore be represented as an 8-digit binary rule table $(t_7t_6t_5t_4t_3t_2t_1t_0)$ where configuration $(000)$ maps to $t_0$, $(001)$ maps to $t_1$, ..., and $(111)$ maps to $(t_7)$. Consequently, there are $2^8=256$ possible transition functions for elementary CA.\\

The Wolfram code, a number between 0 and 255 obtained by converting the binary rule table to decimal, is the standard naming convention for these rules. Rule 110 is particularly notable as it can exhibit class 4 behaviour \cite{wolfram2002} and is Turing complete \cite{cook2004universality}. Figure \ref{fig:rule-110} shows an example progression of a Rule 110 system. Each row of pixels represents the state of the automaton at one snapshot in time with the topmost row representing the randomized initial state. It shows the emergence, interaction, and subsequent dissipation of multiple long-lived impermanent patterns.

\begin{figure}[!h]
\centering
\includegraphics[width=0.9\textwidth]{images/rule-110.png}
\caption{Rule 110 progression with random initialisation \cite{wolfram2002}}
\label{fig:rule-110}
\end{figure}

% \subsection{Morphogenesis}
% Morphogenesis is the process by which a system develops into a particular shape or pattern. 
% Biologically, this is seen in most multicellular organisms which can robustly develop specialised organs and intricate skin patterns without any centralised decision-making.
% Through simple rules encoded in the genome and homeostatic feedback loops enforced through chemical signalling, a tissue knows exactly how to grow and when to stop.\\

% Emulating this behaviour \textit{in silico} can provide great insight into the way self-organising and self-repairing biological systems function.
% Cellular automata are a promising model of computation for artificial life simulation because, much like biological agents, their behaviour follows logically from combining information in their surroundings and internal programming.
% In the context of morphogenesis, we are interested in rules that form stable class 2 patterns from random initial conditions.
% We are also interested in rules that are resistant to noise.
% This is analogous to the behaviour of biological cells which grow into stable configurations and are robust to perturbation and damage during growth.
\raggedbottom
\pagebreak
\section{Evolutionary Algorithms}

Evolutionary algorithms (EAs) are a family of heuristic-based search algorithms for black-box optimisation problems. They are inspired by biological evolution. A population of candidate solutions is initialized and modified through repeated selection, mutation, and recombination. We define an EA formally in Algorithm~\ref{alg:ea}.\\

\begin{algorithm}
  \caption{Schematic Evolutionary Algorithm}\label{alg:ea}
  \begin{algorithmic}
  \Require $S$ - the set of possible chromosome values
  \Ensure $s^* \in S$
  \State $t \gets 0$
  \State $M_0 \gets \mu$ random individuals from $S$
  \While{stopping condition is false}
    \State \Call{Evaluate}{$M_t$}
    \State $P_t \gets$ \Call{SelectParents}{$M_t$}    \Comment{Parents}
    \State $\Lambda_t \gets$ \Call{Recombine}{$P_t$}  \Comment{Children}
    \State $Pmod_t \gets$ \Call{Mutate}{$P_t$}
    \State $\Lambda mod_t \gets$ \Call{Mutate}{$\Lambda_t$}
    \State $M_t+1 \gets$ \Call{SelectPopulation}{$Pmod_t$, $\Lambda mod_t$}
    \State $t \gets t+1$
  \EndWhile
  \State $s^* \gets$ \Call{FindBestCandidate}{$P_t$}
  \end{algorithmic}
\end{algorithm}

The initial selection phase ($\Call{SelectParents}$) uses a objective function, also known as a fitness function, to compare and select the top candidates. Recombination produces a set of children that have similar properties to some subset of the parents. This exploits the cumulative progress of the evolutionary process embedded in the parent candidates. Mutation explores new areas of the search space by perturbing properties of the parents and children. The latter selection phase ($\Call{SelectPopulation}$) produces a new population from the modified parents and children. Population-wide selection criteria can be enforced in this phase. For example, certain parents can be eliminated if they have survived for too many generations or, symmetrically, children can be granted immunity for a particular number of generations.\\

EAs are valued for their broad applicability as they require no information about the constraints or derivative of the objective function. In fact, an explicit representation of the objective function is not even neccessary to run an EA as long as candidates can be compared to each other. Selection pressure can then be introduced in the form of tournament-based elimination.\\

Typically, an EA acts on a population of "chromosomes" which are indirect encodings of candidate solutions. The structure of the chromosome is called the "genotype" and the structure of the corresponding solution is called the "phenotype". In this thesis, the genotype will usually be a set of parameters that characterise the transition function for a particular class of CA. The corresponding phenotype is the cellular automaton with that transition function.\\

