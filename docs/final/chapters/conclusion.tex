\chapter{Conclusions}

This project has shown that evolutionary algorithms can be very powerful tools for tackling inverse problems in the domain of cellular automata. At the outset, we introduced three objectives. We address these in turn.
\begin{enumerate}
    \item \textbf{State optimization for life-like CA}\\
    We built genetic algorithms to learn life-like CA transition functions that randomly generate templates for mazes. With minor post-processing, these become challenging mazes with the desired qualities: long solution paths and many dead ends. This pushes the boundaries on existing work using CA for maze generation\cite{adams2017procedural, adams2018evolving} through a stochastic region merging algorithm that is unbiased with respect to orientation and guarantees playability.
    \item \textbf{Full rule dynamics for life-like CA}\\
    We built genetic algorithms that successfully learned the full rule dynamics of all 100 life-like CA, sampled uniformly across the search space, based on a very small number of observations (in some cases, a single observation). This was a very quick learning process as it required an average of 9 epochs of training on a population of 20 individuals. This represents a traversal of only 0.047\% of the total search space. As the first work to learn full rule dynamics for a class of 2D CA, we establish an important result in this regard.
    \item \textbf{Full rule dynamics for Gray-Scott models}\\
    We further extended these genetic algorithms and implemented an evolutionary strategy method for continuous state CA. When testing on a few examples, these algorithms converge to locally optimal regions which we visualise in the parameter space. This presents an area with great potential for future work.
\end{enumerate}

During this project, an efficient simulation system was built for both life-like and Gray-Scott CA which produces rich media in the form of images and animations. Together with implementations of numerous genetic operators, selection methods, chromosome structures, evaluation metrics, and experiment templates, this forms a cohesive evolutionary algorithm toolkit that can be extended to accommodate other classes of discrete and continuous CA.

\section{Further Work}

This project reveals many potential avenues for future exploration. We briefly discuss some of these here.

\subsection{Procedural Generation}

The use of CA for procedural generation extends to many domains outside of graphics and games. One particular area of interest is network design. Evolutionary algorithms could be used to train CA models of efficient transport networks. This would require some notion of orientation to be embedded within the CA chromosome, thereby making outer-totalistic CA transition functions unsuitable. A better approach may be to consider something similar to the perception vector used by Mordvintsev et al.\cite{mordvintsev2020growing} in their work on neural cellular automata. Here, alongside the number of neighbours, each chromosome would leverage information about the gradient in the $x$ and $y$ directions. Sobel filter convolutions could be used to capture such information.

\subsection{New CA Topologies}

In this work, we focus on CA that exist on square lattices with periodic boundary conditions. This \textit{toroidal} topology simulates an infinite periodic tiling. It would be useful to consider whether the results established in this work extend to other topologies. A simple change would be to consider fixed boundary conditions. A more drastic change would be to consider a lattice of non-uniform cells such as those obtained by performing a Voronoi decomposition. It has been shown neural cellular automata can effectively learn full rule dynamics for Life on Voronoi lattices\cite{grattarola2021learning}. It would be interesting to see if these results can also be achieved with evolutionary computation.

\subsection{Evolutionary Strategies}\todo{why u no do this one??}

When learning on Gray-Scott models, we used evolutionary strategies with modern features such as self-adaption. However, these are not the current state-of-the-art in ES. Using covariance matrix adaptation evolutionary strategies (CMA-ES), which make fewer assumptions about the underlying function, may yield more promising results.

\subsection{Neural Networks}

Neural cellular automata are at the forefront of current research around self-organising systems. Such systems are very successful at complex morphogenesis goals. These systems typically use a CA where the transition function is itself a neural network which is optimized through random observations. Similar techniques may also perform well in domains where evolutionary algorithms appear to fall short, like learning Gray-Scott models.
