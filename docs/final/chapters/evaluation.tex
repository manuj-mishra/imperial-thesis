\chapter{Evaluation}

\section{Ethics}

As a relatively abstract mathematical project, there are no major ethical factors to consider here. However, there are still some legal and societal issues that are worth discussing, especially given that technologies falling in the broad field of self-organising systems, distributed systems, and automated systems can be misused.\\

The only area of legal concern is the infringement of copyright law when selecting training data. To train graph cellular automata, we many use point cloud datasets. These tend to be of physical objects or terrain maps but can, in theory, be derived from any image. When choosing point clouds to train on, we will ensure that they do not fall under copyright restrictions that make them unsuitable for academic use.\\

The only area of societal or professional concern is the general application of cellular automata in distributed technology like swarm robotics. Such systems can have military applications. It could be argued that this research could pave the way for highly distributed swarms of drones or ground robots that are robust to partial damage or perturbation. However, this is unlikely to be a true concern since we only discuss theoretical concepts in this paper with little discussion about real-life applications. Furthermore, this research is only in its preliminary stages and the academic benefits of researching self-organising cellular automata far outweigh the negligible potential contribution that this research could have to improving malicious swarm robotics systems.\\
